\documentclass[11pt,a4paper,oldfontcommands]{memoir}
%\documentclass[]{book}
\usepackage[utf8]{inputenc}
\usepackage[T1]{fontenc}
\usepackage{microtype}
\usepackage[dvips]{graphicx}
\usepackage{xcolor}
\usepackage{times}
\usepackage{lipsum}
\usepackage{amsmath}
\usepackage{blindtext}

\usepackage[
breaklinks=true,colorlinks=true,
%linkcolor=blue,urlcolor=blue,citecolor=blue,% PDF VIEW
linkcolor=black,urlcolor=black,citecolor=black,% PRINT
bookmarks=true,bookmarksopenlevel=2]{hyperref}

\usepackage{geometry}
% PDF VIEW
% \geometry{total={210mm,297mm},
% left=25mm,right=25mm,%
% bindingoffset=0mm, top=25mm,bottom=25mm}
% PRINT
\geometry{total={210mm,297mm},
left=20mm,right=20mm,
bindingoffset=10mm, top=25mm,bottom=25mm}

\OnehalfSpacing
%\linespread{1.3}

%%% CHAPTER'S STYLE
\chapterstyle{bianchi}
%\chapterstyle{ger}
%\chapterstyle{madsen}
%\chapterstyle{ell}
%%% STYLE OF SECTIONS, SUBSECTIONS, AND SUBSUBSECTIONS
\setsecheadstyle{\Large\bfseries\sffamily\raggedright}
\setsubsecheadstyle{\large\bfseries\sffamily\raggedright}
\setsubsubsecheadstyle{\bfseries\sffamily\raggedright}


%%% STYLE OF PAGES NUMBERING
%\pagestyle{companion}\nouppercaseheads 
%\pagestyle{headings}
%\pagestyle{Ruled}
\pagestyle{plain}
\makepagestyle{plain}
\makeevenfoot{plain}{\thepage}{}{}
\makeoddfoot{plain}{}{}{\thepage}
\makeevenhead{plain}{}{}{}
\makeoddhead{plain}{}{}{}


\maxsecnumdepth{subsection} % chapters, sections, and subsections are numbered
\maxtocdepth{subsection} % chapters, sections, and subsections are in the Table of Contents


%%%---%%%---%%%---%%%---%%%---%%%---%%%---%%%---%%%---%%%---%%%---%%%---%%%

\begin{document}

%%%---%%%---%%%---%%%---%%%---%%%---%%%---%%%---%%%---%%%---%%%---%%%---%%%
%   TITLEPAGE
%
%   due to variety of titlepage schemes it is probably better to make titlepage manually
%
%%%---%%%---%%%---%%%---%%%---%%%---%%%---%%%---%%%---%%%---%%%---%%%---%%%
\thispagestyle{empty}

{%%%
\sffamily
\centering
\Large

~\vspace{\fill}

{\huge 
Thesis title: may be long or short
}

\vspace{2.5cm}

{\LARGE
Your name
}

\vspace{3.5cm}

A thesis submitted in partial fulfillment for the\\
degree of Doctor of Philosophy\\[1em]
in the\\[1em]
Faculty Name\\
University Name

\vspace{3.5cm}

Supervisor: Prof. Joe Doe

\vspace{\fill}

May 2013

%%%
}%%%

\cleardoublepage
%%%---%%%---%%%---%%%---%%%---%%%---%%%---%%%---%%%---%%%---%%%---%%%---%%%
%%%---%%%---%%%---%%%---%%%---%%%---%%%---%%%---%%%---%%%---%%%---%%%---%%%


\tableofcontents*


\clearpage

\chapter{Introduction}


This code was created for answer to
\href{http://tex.stackexchange.com/q/237566/2891}{this question on
  TeX.sx}. It shows how to use Mathjax for math rendering, Hypothes.is for
annotating, and custom Javascript libraries for navigational sidebar and
pop-up windows with bibliographic references.
\href{https://github.com/viljamis/Scale}{Scale.css} by Viljami Salminen
is used for responsive web typography.

You can see the result on my
\href{http://michal-h21.github.io/reyman/thesis.html}{Github pages}.
Please note that this is basically a proof of concept, so don't expect
perfect design nor functionality. In particular, the sidebar needs some
polishing, the local TOC can be collapsed and opened with some CSS
animation, it also needs adaption to responsive design, in order to
support small screens. Bibliography pop-up is placed on a fixed
position, it should be more flexible. Fixes for these issues are highly
welcome.

\section{How to compile}\label{how-to-compile}

You may use \href{https://github.com/michal-h21/make4ht}{make4ht}, build
system for \texttt{tex4ht}. Because browsers doesn't support loading of
local web pages with Javascript, you need to place the generated files
on web server. Local server suffices. On Linux, documents are usually
placed in \texttt{/var/www/html/}. This location is predefined in the
build file, \texttt{thesis.mk4}. If you use a different path, modify the
\texttt{outdir} variable.

Compile with

\begin{verbatim}
   make4ht -u -c enhanced.cfg thesis.tex
\end{verbatim}

 if you haven't modify the \texttt{outdir} variable and everything works
 well, you may open \href{http://localhost/reyman/thesis.html}{the
   generated page}


%%%---%%%---%%%---%%%---%%%---%%%---%%%---%%%---%%%---%%%---%%%---%%%---%%%
%%%---%%%---%%%---%%%---%%%---%%%---%%%---%%%---%%%---%%%---%%%---%%%---%%%
\chapter{Math examples}

These examples were copied from 
\href{http://www.maths.adelaide.edu.au/anthony.roberts/LaTeX/Src/maths.tex}{www.maths.adelaide.edu.au/}. 
Math was converted to \texttt{mathml}, \href{https://www.mathjax.org/}{Mathjax}
is used to render it in browsers without \texttt{mathml} support.

\section{Delimiters}

See how the delimiters are of reasonable size in these examples
\[
  \left(a+b\right)\left[1-\frac{b}{a+b}\right]=a\,,
\]
\[
  \sqrt{|xy|}\leq\left|\frac{x+y}{2}\right|,
\]
even when there is no matching delimiter
\[
  \int_a^bu\frac{d^2v}{dx^2}\,dx
  =\left.u\frac{dv}{dx}\right|_a^b
  -\int_a^b\frac{du}{dx}\frac{dv}{dx}\,dx.
\]






\section{Spacing}

Differentials often need a bit of help with their spacing as in
\[
  \iint xy^2\,dx\,dy 
  =\frac{1}{6}x^2y^3,
\]
whereas vector problems often lead to statements such as
\[
  u=\frac{-y}{x^2+y^2}\,,\quad
  v=\frac{x}{x^2+y^2}\,,\quad\text{and}\quad
  w=0\,.
\]







\section{Arrays}

Arrays of mathematics are typeset using one of the matrix environments as 
in
\[
  \begin{bmatrix}
    1 & x & 0 \\
    0 & 1 & -1
  \end{bmatrix}\begin{bmatrix}
    1  \\
    y  \\
    1
  \end{bmatrix}
  =\begin{bmatrix}
    1+xy  \\
    y-1
  \end{bmatrix}.
\]
Case statements use cases:
\[
  |x|=\begin{cases}
    x, & \text{if }x\geq 0\,,  \\
    -x, & \text{if }x< 0\,.
  \end{cases}
\]
Many arrays have lots of dots all over the place as in
\[
  \begin{matrix}
    -2 & 1 & 0 & 0 & \cdots & 0  \\
    1 & -2 & 1 & 0 & \cdots & 0  \\
    0 & 1 & -2 & 1 & \cdots & 0  \\
    0 & 0 & 1 & -2 & \ddots & \vdots \\
    \vdots & \vdots & \vdots & \ddots & \ddots & 1  \\
    0 & 0 & 0 & \cdots & 1 & -2
  \end{matrix}
\]






\section{Equation arrays}

In the flow of a fluid film we may report
\begin{eqnarray}
  u_\alpha & = & \epsilon^2 \kappa_{xxx} 
  \left( y-\frac{1}{2}y^2 \right),
  \label{equ}  \\
  v & = & \epsilon^3 \kappa_{xxx} y\,,
  \label{eqv}  \\
  p & = & \epsilon \kappa_{xx}\,.
  \label{eqp}
\end{eqnarray}
Alternatively, the curl of a vector field $(u,v,w)$ may be written 
with only one equation number:
\begin{eqnarray}
  \omega_1 & = &
  \frac{\partial w}{\partial y}-\frac{\partial v}{\partial z}\,,
  \nonumber  \\
  \omega_2 & = & 
  \frac{\partial u}{\partial z}-\frac{\partial w}{\partial x}\,,
  \label{eqcurl}  \\
  \omega_3 & = & 
  \frac{\partial v}{\partial x}-\frac{\partial u}{\partial y}\,.
  \nonumber
\end{eqnarray}
Whereas a derivation may look like
\begin{eqnarray*}
  (p\wedge q)\vee(p\wedge\neg q) & = & p\wedge(q\vee\neg q)
  \quad\text{by distributive law}  \\
   & = & p\wedge T \quad\text{by excluded middle}  \\
   & = & p \quad\text{by identity}
\end{eqnarray*}






\section{Functions}

Observe that trigonometric and other elementary functions are typeset 
properly, even to the extent of providing a thin space if followed by 
a single letter argument:
\[
  \exp(i\theta)=\cos\theta +i\sin\theta\,,\quad
  \sinh(\log x)=\frac{1}{2}\left( x-\frac{1}{x} \right).
\]
With sub- and super-scripts placed properly on more complicated 
functions,
\[
  \lim_{q\to\infty}\|f(x)\|_q 
  =\max_{x}|f(x)|,
\]
and large operators, such as integrals and
\begin{eqnarray*}
  e^x & = & \sum_{n=0}^\infty \frac{x^n}{n!}
  \quad\text{where }n!=\prod_{i=1}^n i\,,  \\
  \overline{U_\alpha} & = & \bigcap_\alpha U_\alpha\,.
\end{eqnarray*}
In inline mathematics the scripts are correctly placed to the side in 
order to conserve vertical space, as in
\(
  1/(1-x)=\sum_{n=0}^\infty x^n.
\)






\section{Accents}

Mathematical accents are performed by a short command with one 
argument, such as
\[
  \tilde f(\omega)=\frac{1}{2\pi}
  \int_{-\infty}^\infty f(x)e^{-i\omega x}\,dx\,,
\]
or
\[
  \dot{\vec \omega}=\vec r\times\vec I\,.
\]





\section{Command definition}

\newcommand{\Ai}{\operatorname{Ai}} 
The Airy function, $\Ai(x)$, may be incorrectly defined as this 
integral
\[
  \Ai(x)=\int\exp(s^3+isx)\,ds\,.
\]

\newcommand{\D}[2]{\frac{\partial #2}{\partial #1}}
\newcommand{\DD}[2]{\frac{\partial^2 #2}{\partial #1^2}}
\renewcommand{\vec}[1]{\text{\boldmath$#1$}}

This vector identity serves nicely to illustrate two of the new 
commands:
\[
  \vec\nabla\times\vec q
  =\vec i\left(\D yw-\D zv\right)
  +\vec j\left(\D zu-\D xw\right)
  +\vec k\left(\D xv-\D yu\right).
\]




\section{Theorems et al.}

\newtheorem{theorem}{Theorem}
\newtheorem{corollary}[theorem]{Corollary}
\newtheorem{lemma}[theorem]{Lemma}
\newtheorem{definition}[theorem]{Definition}

\begin{definition}[right-angled triangles] \label{def:tri}
A \emph{right-angled triangle} is a triangle whose sides of length~\(a\), \(b\) and~\(c\), in some permutation of order, satisfies \(a^2+b^2=c^2\).
\end{definition}

\begin{lemma} 
The triangle with sides of length~\(3\), \(4\) and~\(5\) is right-angled.
\end{lemma}

This lemma follows from the Definition~\ref{def:tri} as \(3^2+4^2=9+16=25=5^2\).

\begin{theorem}[Pythagorean triplets] \label{thm:py}
Triangles with sides of length \(a=p^2-q^2\), \(b=2pq\) and \(c=p^2+q^2\) are right-angled triangles.
\end{theorem}

Prove this Theorem~\ref{thm:py} by the algebra \(a^2+b^2 =(p^2-q^2)^2+(2pq)^2
=p^4-2p^2q^2+q^4+4p^2q^2
=p^4+2p^2q^2+q^4
=(p^2+q^2)^2 =c^2\).

\Blinddocument

Citation of Einstein paper~\cite{Einstein}. And \cite{Einstein2}

\appendix

\chapter{Additional}
\lipsum[1]

\bibliographystyle{unsrt}
\bibliography{sample}
\makeatletter
\makeatother

\end{document}

