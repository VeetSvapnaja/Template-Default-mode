\chapter{Jadilah Lelucon Untuk Dirimu Sendiri} %361

\bahasa
Pertanyaan pertama:

\english
The first question:

\bahasa
OSHO TERKASIH,\\
TOLONG JELASKAN PERBEDAAN ANTARA SANNYASIN DAN YANG BUKAN, NAMUN HIDUP DENGAN KOMITMEN YANG MENDALAM TERHADAP KEBENARAN.

\english
BELOVED OSHO,\\
PLEASE EXPLAIN THE DIFFERENCE BETWEEN A SANNYASIN AND ONE WHO IS NOT, YET LIVES WITH A DEEP COMMITMENT TO TRUTH.

\bahasa
Lynne Stevens, apakah engkau tahu apa itu kebenaran? Jika tidak, bagaimana dapat ada komitmen? Komitmen hanya mungkin jika engkau tahu. Sannyasin adalah orang yang tahu bahwa dia tidak tahu, sannyasin adalah orang yang komitmennya tidak untuk kebenaran, namun untuk penyelidikan kebenaran. Dan penyelidikannya adalah mungkin hanya dengan seseorang yang tahu, seseorang yang telah tiba. Sannyasin adalah seseorang yang berkomitmen pada orang tersebut, atau pada bukan-pribadi, yang berada didekatnya dia merasakan getaran kebenaran, getaran keaslian.

\english
Lynne Stevens, do you know what truth is? Otherwise, how can there be a commitment? Commitment is possible only if you know. The sannyasin is one who knows that he knows not, the sannyasin is one whose commitment is not to truth but to the inquiry into truth. And the inquiry is possible only with someone who knows, who has arrived. The sannyasin is one who is committed to the person, or to the no-person, around whom he feels the vibe of truth, the vibe of authenticity.

\bahasa
Lynne Stevens, komitmenmu terhadap kebenaran hanyalah sebuah gagasan. Kebenaranmu hanyalah sebuah kata, sebuah perjalanan pikiran. Jika engkau ingin menjadikannya ziarah yang sebenarnya engkau harus menjadi murid - dan menjadi murid adalah menjadi sannyasin.

\english
Lynne Stevens, your commitment to truth is just an idea. Your truth is just a word, a mind trip. If you want to make it a real pilgrimage you will have to be a disciple -- and to be a disciple is to be a sannyasin.

\bahasa
Menjadi murid berarti siap untuk belajar, siap untuk masuk kedalam yang tidak diketahui dengan seseorang yang pernah berada di dalamnya. Sendiri, sangat jarang seseorang telah mencapai kebenaran. Bukan berarti hal itu tidak terjadi - sendiri, juga, hal itu telah terjadi, tapi sangat jarang, hanya pengecualian; jika tidak, seseorang harus belajar dalam persekutuan dengan seorang master.

\english
To be a disciple means to be ready to learn, ready to go into the unknown with someone who has been in it. Alone, very rarely one has attained to truth. Not that it has not happened -- alone, also, it has happened, but very rarely, just an exception; otherwise one has to learn in communion with a master.

\bahasa
Lalu juga, itu tidak mudah terjadi. Itu adalah perjalanan yang sulit. Menjatuhkan kemelekatan terhadap yang diketahui tidaklah mudah. Itulah keseluruhan investasi kita, itulah keseluruhan identitas kita. Menjatuhkan kemelekatan yang diketahui adalah menjatuhkan ego, melakukan semacam bunuh diri spiritual; sendirian, engkau tidak akan dapat melakukannya. Kecuali engkau melihat seseorang yang telah melakukan bunuh diri itu dan masih ada -- sebenarnya untuk pertama kalinya ada.... Engkau harus melihat ke dalam mata yang telah melihat kebenaran, dan sekilas kebenaran akan tertangkap melalui mata itu. . Engkau harus berpegangan tangan dengan seseorang yang telah mengetahui, menerima kehangatannya dan cintanya ... dan yang tidak diketahui akan mulai mengalir ke dalam dirimu.

\english
Then too, it does not happen easily. It is an arduous journey. Dropping the clinging to the known is not easy. That is our whole investment, that is our whole identity. Dropping the clinging to the known is dropping the ego, is committing a kind of spiritual suicide; alone, you will not be able to do it. Unless you see somebody who has committed that suicide and still is -- in fact for the first time is.... You will have to look into those eyes which have seen truth, and a glimpse of the truth will be caught through those eyes. You will have to hold hands with someone who has known, receive the warmth and the love... and the unknown will start flowing into you.

\bahasa
Itulah artinya bersama dengan seorang master, untuk menjadi murid. Jika engkau benar-benar berkomitmen terhadap kebenaran, engkau pasti akan menjadi sannyasin. Jika komitmenmu terhadap kebenaran adalah sebuah penyelidikan maka engkau harus mempelajari cara belajar. Dan hal pertama yang harus dipelajari adalah untuk berserah-diri, untuk percaya, untuk mencintai.

\english
That's what it means to be with a master, to be a disciple. If you are really committed to truth you are bound to become a sannyasin. If your commitment to truth is an inquiry then you will have to learn the ways of learning. And the first thing to learn is to surrender, to trust, to love.

\bahasa
Sannyasin adalah orang yang telah jatuh cinta pada seseorang, atau bukan seorang pribadi, di mana dia merasakan perasaan yang mendalam: "Ya, hal itu telah terjadi di sini." Untuk bersama seseorang yang dikenal menular -- dan kebenaran tidak diajarkan, kebenaran itu ditangkap.

\english
The sannyasin is one who has fallen in love with a person, or a no-person, where he feels a gut feeling: "Yes, it has happened here." To be with someone who has known is contagious -- and truth is not taught, it is caught.

\bahasa
Kebenaranmu hanyalah sebuah gagasan di dalam pikiranmu - mungkin sebuah penyelidikan filosofis, tapi sebuah penyelidikan filosofis tidak akan membantu. Itu harus menjadi eksistensial, engkau harus memberikan bukti dalam hidupmu bahwa engkau benar-benar berkomitmen. Jika tidak, engkau dapat terus memainkan permainan kata-kata, permainan teori, sistem pemikiran yang indah - dan ada ribuan. Engkau juga dapat membuat sistem pemikiran pribadimu sendiri, dan engkau akan menganggap ini adalah kebenaran.

\english
Your truth is nothing but an idea in your mind -- maybe a philosophical inquiry, but a philosophical inquiry is not going to help. It has to become existential, you have to give proofs in your life that you are really committed. Otherwise you can go on playing the game of words, beautiful games of theories, systems of thought -- and there are thousands. You can also make a private system of thought of your own, and you will think this is truth.

\bahasa
Kebenaran bukanlah pembuatanmu, kebenaran tidak ada hubungannya dengan pikiranmu. Kebenaran terjadi, dan itu terjadi hanya jika engkau memiliki sebuah tanpa-pikiran. Tapi bagaimana engkau akan menjadi tanpa-pikiran? Dengan caramu sendiri engkau akan tetap menjadi si pikiran. Engkau mungkin berpikir tentang tanpa-pikiran, engkau dapat berfilsafat tentang tanpa-pikiran, engkau dapat membaca kitab suci tentang tanpa-pikiran, namun engkau akan tetap menjadi si pikiran. Dengan caramu sendiri, menyelidiki dan mencari, egomu akan merasa sangat senang - tapi itu adalah penghalang. Analoginya seperti menarik dirimu dengan benang kusutmu sendiri.

\english
Truth is not of your making, truth has nothing to do with your mind. Truth happens, and it happens only when you have become a no-mind. But how are you going to become a no-mind? On your own you will remain the mind. You may think about the no-mind, you may philosophize about the no-mind, you may read the scriptures about no-mind, but you will remain a mind. On your own, seeking and searching, your ego will feel very good -- but that is the barrier. It is like pulling yourself up by your own bootstraps.

\bahasa
Jika di suatu tempat engkau menemukan bantuan tersedia, jangan lewatkan itu - karena kesempatan itu langka, buddhafield jarang terjadi. Hanya sekali-sekali, entah di mana, seorang buddha muncul, bodhichitta terjadi. Maka jangan lewatkan kesempatan itu. Jika komitmenmu benar-benar menuju pada kebenaran, engkau tidak dapat menghindari menjadi sannyasin. Hal ini tak terelakkan, karena tanpa pikiran dipelajari hanya dengan duduk di sisi seorang tanpa pikiran.

\english
If somewhere you find help is available, don't miss it -- because the opportunity is rare, the buddhafield is rare. Only once in a while, somewhere, a buddha arises, a bodhichitta happens. Then don't miss the opportunity. If your commitment is really towards truth, you cannot avoid becoming a sannyasin. It is inevitable, because no-mind is learned only by sitting by the side of a no-mind.

\bahasa
Jika engkau duduk di sisiku, perlahan pelan-pelan pikiranmu akan mulai menghilang seperti kabut pagi. Perlahan perlahan keheningan akan mulai menembusmu -- bukan tindakanmu, tapi datang dengan sendirinya. Keheningan akan menyelimutimu.

\english
If you sit by my side, slowly slowly your mind will start dispersing like the morning mist. Slowly slowly a silence will start penetrating you -- not of your doing, but coming on its own. A stillness will pervade you.

\bahasa
Dan saat engkau benar-benar diam, bahkan pikiran yang bergerak di dalam dirimu, itulah momen iluminasi Untuk pertama kalinya engkau memiliki sekilas kebenaran - bukan gagasan akan kebenaran, tapi kebenaran itu sendiri

\english
And the moment you are utterly still, not even a thought moving inside you, that is the moment of illumination. For the first time you have a glimpse of truth -- not the idea of truth, but truth itself.

\bahasa
Pertanyaan Kedua:

\english
The second question:

\bahasa
OSHO Terkasih,\\
PERASAANKU MENGATAKAN BAHWA SAMPAI AKU MENGENALMU, AKU TIDAK DAPAT PERCAYA, NAMUN ENGKAU MENGATAKAN SAMPAI AKU MEMPERCAYAIMU, AKU TIDAK DAPAT MENGENALMU. APA YANG HARUS DILAKUKAN?

\english
BELOVED OSHO,\\
MY FEELINGS TELL ME THAT UNTIL I KNOW YOU, I CAN'T TRUST. AND YET YOU SAY UNTIL I TRUST YOU, I CANNOT KNOW YOU. WHAT TO DO?

\bahasa
William, ada dua jenis pengetahuan. Yang satu adalah dari kejauhan: engkau tetap menyendiri, engkau tetap menjadi seorang pengamat, penonton. Itulah pengetahuan secara ilmiah; engkau tidak perlu terlibat di dalamnya, sebenarnya engkau seharusnya tidak terlibat. Engkau harus sangat objektif, engkau seharusnuya tidak membiarkan subjektivitasmu mencampuri pengamatanmu. Engkau seharusnya berada di sana seperti seorang pengamat mekanis. Engkau seharusnya tidak menjadi manusia, engkau seharusnya hanya menjadi sebuah komputer.

\english
William, there are two kinds of knowing. One is from a distance: you remain aloof, you remain an observer, a spectator. That's what scientific knowing is; you need not get involved in it, in fact you should not get involved. You should be very objective, you should not allow your subjectivity to interfere with your observation. You should simply be there like a mechanical watcher. You should not be a human being, you should be just a computer.

\bahasa
Dan ini pasti, bahwa cepat atau lambat sains akan diambil alih oleh komputer, robot, karena mereka akan lebih ilmiah. Tidak akan ada subjektivitas di dalamnya, mereka hanya akan melihat fakta. Fakta tidak akan dicampuri dengan cara apapun, itu akan tetap benar-benar objektif.

\english
And this is certain, that sooner or later science is going to be taken over by computers, robots, because they will be more scientific. There will be no subjectivity in them, they will simply see the fact. The fact will not be interfered with in any way, it will remain utterly objective.

\bahasa
Itulah cara sains -- mengetahui dari kejauhan, menjaga jarak, terpisah. Begitulah cara ilmuwan akan mengetahui sekuntum bunga mawar, begitulah cara ilmuwan akan mengetahui matahari terbenam, begitulah cara ilmuwan akan mengetahui keindahan seorang wanita atau seorang pria.

\english
That is the way of science -- knowing from a distance, keeping aloof, detached. That's how the scientist will know a rose flower, that's how the scientist will know the sunset, that's how the scientist will know the beauty of a woman or a man.

\bahasa
Tapi masalahnya adalah, sesuatu yang esensial pasti akan dilewatkan, sesuatu yang sangat fundamental, sesuatu yang merupakan inti dari semuanya. Ilmuwan dapat mengetahui bunga mawar -- dia dapat tahu itu terbentuk dari apa, dia dapat mengetahui semua bahan kimia,dan sebagainya, tapi dia tidak akan pernah tahu keindahannya. Dia akan tetap buta terhadap keindahan; pendekatannya, metodologinya, menghalanginya.

\english
But the problem is, something essential is bound to be missed, something very
fundamental, something which is the core of the whole thing. The scientist can know the roseflower -- he can know of what it is constituted, he can know all the chemicals, etcetera, but he will never know the beauty of it. He will remain blind to the beauty; his very approach, his methodology, prohibits him.

\bahasa
Jika engkau terpisah engkau tidak dapat mengenal keindahan. Keindahan hanya diketahui saat engkau jatuh dalam hubungan yang erat, saat pengamat menjadi yang teramati, bila tidak ada dinding di antara, ketika setiap dinding telah berubah menjadi jembatan. Bila ada semacam pencairan, ketika engkau menjadi bunga dan bunga menjadi dirimu, maka ada jenis pengetahuan yang sama sekali berbeda - seperti yang diketahui seorang penyair. Dia akan tahu keindahan, dia tidak akan tahu bahan kimianya. Dia tidak akan tahu bunga yang objektif, dia akan tahu sesuatu yang jauh lebih dalam. Dia akan mengetahui spiritualitas dari bunga, jiwa dari bunga.

\english
If you are detached you cannot know beauty. Beauty is known only when you fall en rapport, when the observer becomes the observed, when there is no wall between, when every wall has been transformed into a bridge. When there is a kind of melting, when you become the flower and the flower becomes you, then there is a totally different kind of knowing -- the way a poet knows. He will know beauty, he will not know the chemicals. He will not know the objective flower, he will know something far deeper. He will know the spirituality of the flower, the spirit of the flower.

\bahasa
Dan para mistik, pengetahuannya adalah bentuk tertinggi dari pengetahuan puitis, bentuk akhir dari pengetahuan puitis. Penyair hanya ada untuk beberapa saat. Terkadang dia seorang penyair, dia bertemu, dia bercampur, menyatu menjadi bunga; terkadang ia menjadi pengamat yang terpisah. Oleh karena itu puisi adalah semacam campuran dari kedua pengetahuan tersebut.

\english
And the mystic, his knowing is the highest form of poetic knowing, the ultimate form of poetic knowing. The poet is there only for moments. Sometimes he is a poet, he meets, he mingles, merges into the flower; sometimes he becomes a detached observer. Hence poetry is a kind of mixture of both the knowledges.

\bahasa
Pengetahuan ilmiah murni objektif, pengetahuan mistis murni subjektif, pengetahuan puitis adalah di antara keduanya, campuran keduanya - sedikit sains, sedikit agama. Tapi pengetahuan dasar dapat dibagi menjadi dua, ilmiah dan mistik.Sekarang tergantung padamu, dengan cara apa engkau ingin mengetahui aku.

\english
Scientific knowledge is purely objective, mystic knowing is purely subjective, poetic knowing is between the two, a mixture of both -- a little bit of science, a little bit of religion. But basic knowing can be divided in two, the scientific and the mystic. Now it depends on you, in what way you want to know me.

\bahasa
Engkau berkata: "Perasaanku mengatakan kepadaku bahwa sampai aku mengenalmu, aku tidak dapat mempercayaimu."

\english
You say: "My feelings tell me that until I know you I can't trust you."

\bahasa
Ini bukan perasaan - disitu engkau salah paham dengan dirimu sendiri. Ini adalah pemikiran, ini tidak dapat menjadi perasaan; Itu adalah kesalahpahaman belaka. Inilah cara pemikiran berbicara. Pemikiran selalu berkata, "Hati-hati, awas, bergeraklah secara logis" -- dan tentu saja ini sangat logis: bagaimana engkau dapat mempercayaiku jika engkau tidak mengenalku? Ini adalah pernyataan logis. Ini bukan pernyataan dari firasatmu; itu tidak mungkin, karena firasat sangat tidak masuk akal. Firasat akan mengatakan kepadamu sama seperti yang aku katakan: percaya dan engkau akan tahu.

\english
These are not feelings -- there you are misunderstanding yourself. These are thoughts, these cannot be feelings; that's a sheer misunderstanding. This is the way thoughts speak. Thoughts always say, "Be careful, cautious, move logically" -- and of course this is very logical: how can you trust me if you don't know me? It is a logical statement. It is not a statement from your gut level; it cannot be, because gut feelings are very illogical. Gut feelings will say to you the same as I am saying: trust and you will know.

\bahasa
Jadi hal pertama yang dapat dikatakan adalah: ini bukan perasaanmu, ini adalah pemikiranmu. Engkau perhatikan lagi, engkau masuk ke yang disebut perasaan-perasaan ini lagi, dan engkau akan mendapati bahwa perasaan-perasaan itu tidak berasal dari hati, perasaan-perasaan itu berasal dari kepala. Kepala berkata, "Pertama tahu, kemudian percayalah."

\english
So the first thing to be said is: these are not your feelings, these are your thoughts. You watch again, you go into these so-called feelings again, and you will find they are not coming from the heart, they are coming from the head. The head says, "First know, then trust."

\bahasa
Dan ini adalah sebuah strategi yang hebat, jika engkau percaya pada kepala dan pendikteannya - "Pertama tahu,kemudian engkau dapat percaya. " Maka engkau tidak akan pernah percaya, karena tahu tidak dapat terjadi tanpa kepercayaan, pengetahuan mistis tidak dapat terjadi tanpa kepercayaan. Pengetahuan ilmiah itu mungkin, tapi pengetahuan ilmiah tidak berlaku disiini.

\english
And this is a great strategy, if you believe in the head and its dictation -- "First know, then you can trust." Then you will never trust, because knowing cannot happen without trust, mystic knowing cannot happen without trust. Scientific knowing is possible, but scientific knowing is not applicable here.

\bahasa
Engkau dapat mengenalku secara ilmiah. Dokterku datang untuk memeriksa tubuhku; dia mengenalku dengan cara tertentu. Engkau tidak mengenalku dengan cara itu, engkau mengenalku dengan cara yang sangat berbeda. Dokterku takut datang untuk mendengarkanku, karena dia tidak ingin kehilangan seorang pasien. Jika dia mendengarkan aku, maka aku akan menjadi dokter dan dia akan menjadi pasien! Dia datang dan dia bergegas untuk melarikan diri

\english
You can know me scientifically. My doctor comes to examine my body; he knows me in a way. You don't know me in that way, you know me in a totally different way. My doctor is afraid to come to listen to me, because he does not want to lose a patient. If he listens to me, then I will be the doctor and he will be the patient! He comes and he is in a hurry to escape.

\bahasa
Pernah terjadi bahwa dia memegang tanganku -- Aku mengalami masalah dengan ibu jariku -- dan sesuatu terjadi pada dirinya yang tidak ilmiah. Di luar ruangan, dia memberitahu Vivek, "Dia adalah Tuhan, dia adalah Tuhan!" -- Tapi sejak saat itu aku belum melihatnya, dia menghilang. Sesuatu yang tidak saintifik , sesuatu yang bukan dari kepalanya .... Dia merasakanku tuk sesaat tapi menjadi ketakutan.

\english
Once it happened that he was holding my hand -- I had some trouble with my thumb -- and something happened to him which was not scientific. Outside the room, he told Vivek, "He is God, he IS God!" -- but since then I have not seen him, he has simply disappeared. Something nonscientific, something which was not of the head.... He felt me for a moment but became frightened.

\bahasa
Amati. Jika kepalamu mengatakan hal-hal ini, ini bukan perasaan. Perasaan tidak dapat mengatakan hal ini, karena ini bukan bahasa perasaan. Perasaan berkata: "Jatuh cinta, dan kemudian engkau akan tahu." Pemikiran mengatakan: "ragulah, selidi, pastikan. Bila semuanya benar-benar terbukti dan engkau yakin, yakin secara rasional, maka engkau dapat percaya." Dan logika nampak sangat, sangat jernih, nampaknya tidak ada trik di dalamnya. Tapi itu ada! Triknya adalah bahwa melalui pengetahuan ilmiah engkau tidak dapat mengetahui misteri yang engkau hadapi, engkau tidak dapat mengetahui puisi yang mencurahi dirimu, engkau tidak dapat melihat keindahan dan anugerah yang tersedia untukmu.

\english
Watch. If your head is saying these things, these are not feelings. Feelings cannot say these things, because this is not the language of feelings. Feelings say: "Fall in love, and then you will know." Thoughts say: "Doubt, inquire, make certain. When everything is absolutely proved and you are convinced, rationally convinced, then you can trust." And the logic appears very, very clean, there seems to be no trick in it. There is! The trick is that through scientific knowing you cannot know the mystery that is confronting you, you cannot know the poetry that is showering on you, you cannot see the beauty and the grace that is available to you.

\bahasa
Engkau akan melihat tubuhku, engkau akan mendengarkan kata-kataku, tapi engkau akan melewatkan keheningan-keheninganku. Dan mereka adalah pesan-pesan nyataku. Engkau akan dapat melihatku saat aku muncul di permukaan, tapi engkau tidak akan dapat menembusku saat aku berada di pusat.

\english
You will see my body, you will listen to my words, but you will miss my silences. And they are my real messages. You will be able to see me as I appear on the surface, but you will not be able to penetrate into me as I am at the center.

\bahasa
Mengetahui lingkar luar itu mungkin secara ilmiah, tapi dengan mengetahui lingkar luar seseorang cinta tidak muncul. Dan hubungan antara murid dan guru adalah klimaks dari cinta, puncak tertinggi dari cinta. Cinta tidak dapat pergi lebih tinggi dari itu; itulah cinta tertinggi.

\english
Knowing the circumference is possible scientifically, but by knowing the circumference of a person love does not arise. And the relationship between the disciple and the master is the crescendo of love, the highest peak of love. Love cannot go higher than that; that is the ultimate in love.

\bahasa
Inilah pemikiran-pemikiranmu, bukan perasaan-perasaan. Dan jika engkau mendengarkan pemikiran-pemikiran engkau tidak dapat memiliki persekutuan denganku. Engkau akan mendengarkan kata-kataku, engkau akan mendengarkan argumen-argumenku, engkau akan menjadi lebih berpengetahuan, engkau akan menjadi benar-benar puas karena engkau memiliki sesuatu padamu. Dan semua itu adalah omong kosong. Kata-kata yang telah engkau kumpulkan, pengetahuan yang telah engkau kumpulkan, sama sekali tidak berguna.

\english
These are your thoughts, not feelings. And if you listen to thoughts you cannot have any communion with me. You will listen to my words, you will listen to my arguments, you will become more knowledgeable, you will go perfectly satisfied that you have something with you. And all that is nonsense. Those words that you have accumulated, the knowledge that you have gathered, are of no use at all.

\bahasa
Ini bukan pertanyaan untuk mengumpulkan informasi di sini, ini adalah pertanyaan untuk menyerap jiwa; satu-satunya cara adalah percaya. Hanya melalui kepercayaan sehingga mengetahui terjadi.

\english
It is not a question of gathering information here, it is a question of imbibing the spirit; the only way is to trust. It is only through trust that knowing happens.

\bahasa
Sains menggunakan keraguan sebagai metodenya, agama menggunakan kepercayaan sebagai metodenya. Itulah perbedaan mereka yang mendasar. Keraguan tidak relevan dalam dunia cinta, sama seperti kepercayaan tidak relevan dalam dunia benda. Dalam dunia Aku/itu, hanya keraguan yang berarti: engkau tidak dapat mempercayai sesuatu, ilmuwan tidak dapat hanya duduk di sana dalam kepercayaan menunggu sesuatu terjadi. Tidak ada yang akan terjadi. Dia harus meragukan, bertanya, menyelidiki, membedah. Dia harus menggunakan pikirannya, logikanya, maka hanya beberapa kesimpulan yang bisa didapat.

\english
Science uses doubt as its method, religion uses trust as its method. That is their fundamental difference. Doubt is irrelevant in the world of love, just as trust is irrelevant in the world of things. In the world of I/it, only doubt is significant: you cannot trust things, the scientist cannot just sit there in trust waiting for something to happen. Nothing will happen. He has to doubt, inquire, investigate, dissect. He has to use his mind, his logic, then only some conclusions can be arrived at.

\bahasa
Dan kesimpulan-kesimpulan tersebut akan selalu menjadi perkiraan, mereka akan selalu menjadi terkondisi, karena di masa depan lebih banyak fakta dapat diketahui dan keseluruhannya harus diubah lagi. Mereka tidak dapat menjadi mutlak.

\english
And those conclusions will always remain approximate, they will always remain
conditional, because in the future more facts may be known and the whole thing would have to be changed again. They cannot be categorical.

\bahasa
Jadi kepercayaan bukanlah intinya, itu tidak pernah muncul dalam dunia sains; keraguan tetap menjadi dasar. Jika terkadang engkau sampai pada sebuah kesimpulan, kesimpulan itu tidak menjadi kepercayaanmu, tidak menjadi imanmu. Itu tetap merupakan hipotesis.

\english
So trust is not the point, it never arises in the world of science; doubt remains the base. If sometimes you come to a conclusion, the conclusion does not become your trust, does not become your faith. It remains an hypothesis.

\bahasa
Hipotesis berarti bahwa sampai sekarang apapun yang telah diketahui mendukung teori ini. Itu hanya sampai sekarang; kita tidak dapat mengatakan apa-apa tentang hari esok. Besok lebih banyak fakta dapat diketahui, dan pastinya ketika lebih banyak fakta akan diketahui hipotesisnya harus disesuaikan, dan teorinya harus diubah.

\english
An hypothesis means that up to now whatsoever has been known supports this theory. It is only up to now; we can't say anything about tomorrow. Tomorrow more facts may be known, and certainly when more facts will be known the hypothesis will have to be adjusted, and the theory would have to be changed.

\bahasa
Sains terus berubah setiap hari; itu sementara, itu hidup di dunia waktu, karena pikiran adalah waktu. Pikiran tidak dapat hidup tanpa waktu; pikiran adalah sesaat, sementara.

\english
Science goes on changing every day; it is temporal, it lives in the world of time, because mind is time. Mind cannot live without time; mind is momentary, temporal.

\bahasa
Dunia agama berfungsi dalam dimensi yang sama sekali berbeda, pada tingkat yang berbeda. Itu dimulai dengan kepercayaan, cinta, maka jenis pengetahuan yang sama sekali berbeda terjadi.

\english
The world of religion functions in a totally different dimension, on a different level. It begins in trust, in love, then a totally different kind of knowing happens.

\bahasa
Ketika engkau mencintai wanita, engkau mengenalnya. Engkau mengenalnya bukan seperti ginekolog mengenalnya, engkau mengenalnya dengan cara yang sangat berbeda. Engkau tidak tahu fisiologinya, engkau tidak tahu keberadaan materinya, tapi engkau tahu kehadiran spiritualnya. Cinta, hanya cinta, mampu mengetahui kehadiran spiritual. Engkau jatuh cinta tidak dengan tubuh fisik, engkau jatuh cinta dengan kehadiran spiritual seseorang. Tapi itu hanya tersedia dalam kepercayaan. Dalam sains, kepercayaan sama sekali tidak berguna. Dalam agama, keraguan sama sekali tidak berguna.

\english
When you love a woman, you know her. You know her not as the gynecologist knows
her, you know her in a totally different way. You don't know her physiology, you don't know her material existence, but you know her spiritual presence. Love, only love, is capable of knowing the spiritual presence. You fall in love not with the physical body, you fall in love with the spiritual presence of a person. But that is available only in trust. In science, trust is utterly useless. In religion, doubt is utterly useless.

\bahasa
Jadi, William, terserah dirimu. Jika engkau datang ke sini untuk mempelajari apa yang terjadi di sini secara ilmiah, maka engkau dipersilahkan. Engkau dapat berjalan sesuai dengan pemikiran-pemikiranmu sendiri -- tolong jangan panggil mereka perasaan-persaaan. Engkau dapat terus sesuai kepalamu -- jangan sebut itu hatimu, itu bukan. Engkau dipersilahkan: berada disini, belajar, mengamati, sampai pada kesimpulan-kesimpulan tertentu -- namun kesimpulan-kesimpulan itu tetap merupakan hipotesis.

\english
So, William, it is up to you. If you have come here to study what is happening here scientifically, then you are welcome. You can go according to your own thoughts -- don't call them feelings, please. You can go on according to your head -- don't call it your heart, it is not. You are welcome: be here, study, observe, come to certain conclusions -- but they will remain hypotheses.

\bahasa
Tapi jika engkau datang untuk ditransformasikan, tidak untuk diinformasikan saja, maka engkau harus mengerti bahwa ada pintu yang berbeda. Dan pintu itu adalah kepercayaan. Kepercayaan itu fenomena yang tidak masuk akal, tidak masuk akal secara logis. Itulah mengapa logika selalu mengatakan cinta itu buta, meski cinta memiliki pandangan sendiri, jauh lebih dalam lagi ... tetap saja, untuk logika cinta itu buta.

\english
But if you have come to be transformed, not to be informed only, then you will have to understand that there is a different door. And that door is trust. Trust is an absurd phenomenon, logically absurd. That's why logic always says love is blind, although love has its own eyes, far more deep-going... still, to logic it is blind.

\bahasa
Logika menertawakan cinta, dan cinta tersenyum dengan sadar pada kebodohan logika.

\english
Logic ridicules love, and love smiles knowingly at the whole foolishness of logic.

\bahasa
Jika engkau datang ke sini dengan sebuah pendekatan logis ... pelajari, amati, datang ke beberapa kesimpulan-kesimpulan, tapi mereka tidak akan mengubahmu; itu yang harus engkau sadari. Jika engkau datang untuk ditransformasikan, maka jatuh cinta lah. Lalu lupakan kepala, lalu biarkan ada kontak hati-ke-hati, jiwa-ke-jiwa. Maka tidak perlu untuk terlalu peduli dengan apa yang engkau lihat, seluruh kepedulianmu haruslah dengan apa yang engkau rasakan. Maka engkau seharusnya tidak terlalu peduli dalam mengumpulkan informasi, tapi lebih merayakannya bersamaku. Maka jangan banyak meperhatikan apa yang aku katakan, perhatikanlah aku. Dengarkan keheninganku, jeda, celah, interval -- aku lebih ada di sana. Maka engkau akan menyadari dunia yang sama sekali berbeda yang ada di sini, buddhafield. Itu adalah medan energi; engkau harus terbuka dan rentan terhadapnya, hanya dengan begitu itu dapat menembusmu, menguasaimu, meliputimu.

\english
If you have come here with a logical approach... study, observe, come to some
conclusions, but they are not going to transform you; that much you must be aware of. If you have come to be transformed, then fall in love. Then forget the head, then let there be a contact heart-to-heart, spirit-to-spirit. Then there is no need to be too much concerned with what you see, your whole concern should be with what you feel. Then you should not be too much concerned in collecting information, but being in celebration with me. Then don't take much note of what I say, take note of what I am. Listen to my silences, the pauses, the gaps, the intervals -- I am more there. Then you will become aware of a totally different world existing here, the buddhafield. It is an energy field; you have to be open and vulnerable to it, only then it can permeate you, pervade you, overwhelm you.

\bahasa
Pertanyaan ketiga:

\english
The third question:

\bahasa
OSHO TERKASIH,\\
ENGKAU BERKATA DI HARI YANG LAIN BAHWA TIDAK ADA YANG TERTARIK LAGI DALAM PERTANYAAN SEPERTI "SIAPA YANG MENCIPTAKAN ALAM SEMESTA?" NAMUN MAJALAH TIME EDISI BARU-BARU INI MENGKHUSUKAN CUKUP BANYAK RUANG UNTUK SEBUAH ARTIKEL BERJUDUL "PADA MULANYA: TUHAN DAN SAINS."

TEMA DASAR ARTIKEL TERSEBUT ADALAH BAHWA SAINS DAN AGAMA TELAH DIPERSATUKAN OLEH TEORI PENCIPTAAN "BIG BANG" DI MANA ALAM SEMESTA DIDUGA TERBENTUK MELALUI LEDAKAN BOLA API YANG LUAR BIASA, LIMA BELAS ATAU DUA PULUH JUTA TAHUN YANG LALU.

MAJALAN TIME MENGATAKAN BAHWA INI TERDENGAR SANGAT MIRIP DENGAN CERITA YANG TELAH LAMA DICERITAKAN DALAM PERJANJIAN LAMA, YAITU BAHWA ALAM SEMESTA DIMULAI DALAM SATU BENTUK PENCIPTAAN DALAM SEKEJAP MATA.

APA YANG SALAH DENGAN HIPOTESIS YANG MENGATAKAN BAHWA ALAM SEMESTA DICIPTAKAN, BAHWA ITU MEMILIKI AWAL? DAN MENGAPA ENGKAU MENYATAKAN BAHWA ITU TIDAK? BUKANKAH ITU LANGKAH KE ARAH YANG BENAR SAAT SAINS DAN AGAMA SALING SETUJU?

\english
BELOVED OSHO,\\
YOU SAID THE OTHER DAY THAT NO ONE IS INTERESTED ANY MORE IN QUESTIONS LIKE "WHO CREATED THE UNIVERSE?" BUT A RECENT EDITION OF TIME MAGAZINE DEVOTED CONSIDERABLE SPACE TO AN ARTICLE ENTITLED "IN THE BEGINNING: GOD AND SCIENCE."

THE BASIC THEME OF THE ARTICLE WAS THAT SCIENCE AND RELIGION HAVE BEEN BROUGHT CLOSE TOGETHER BY THE "BIG BANG" THEORY OF CREATION IN WHICH THE UNIVERSE IS SUPPOSED TO HAVE COME INTO BEING THROUGH A VAST FIREBALL EXPLOSION, FIFTEEN OR TWENTY MILLION YEARS AGO.

TIME SAYS THAT THIS SOUNDS VERY MUCH LIKE THE STORY WHICH THE OLD TESTAMENT HAS BEEN TELLING ALL ALONG, NAMELY THAT THE UNIVERSE BEGAN IN A SINGLE FLASHING ACT OF CREATION.

WHAT IS WRONG WITH THE HYPOTHESIS THAT THE UNIVERSE WAS CREATED, THAT IT HAD A BEGINNING? AND WHY DO YOU ASSERT THAT IT DID NOT? IS IT NOT A STEP IN THE RIGHT DIRECTION WHEN SCIENCE AND RELIGION AGREE?

\bahasa
Subhuti, hal pertama yang harus diingat adalah, selama tiga ratus tahun agama telah kehilangan wilayahnya secara terus menerus. Pertama, agama mencoba menghancurkan sains. Tidak dapat melakukannya -- karena engkau tidak dapat menghancurkan kebenaran, dan sains lebih benar, sejauh menyangkut dunia obyektif, daripada agama. Sebenarnya agama tidak memiliki wewenang untuk mengatakan apapun tentang dunia obyektif.

\english
Subhuti, the first thing to remember is, for three hundred years religion has been losing its territory continuously. First, religion tried to destroy science. It was unable to do it -- because you cannot destroy truth, and science was truer, as far as the objective world is concerned, than religion. In fact religion has no authority to say anything about the objective world.

\bahasa
Ketika engkau sakit engkau pergi ke dokter, engkau tidak pergi ke penyair. Penyair tidak memiliki wewenang; dia mungkin penyair hebat tapi itu tidak relevan saat engkau sakit. Dia mungkin penyair hebat, tapi ketika ada yang tidak beres di kamar mandimu, engkau tidak memanggilnya, engkau memanggil tukang ledeng. Tukang ledeng mungkin bukan penyair sama sekali, tapi tukang ledeng itu relevan di sana. Engkau tidak memanggil Albert Einstein -- dia mungkin seorang fisikawan yang hebat, tapi apa yang dia ketahui tentang pekerjaan tukang ledeng?

\english
When you are ill you go to the physician, you don't go to the poet. The poet has no authority; he may be a great poet but that is irrelevant when you are ill. He may be a great poet, but when something goes wrong in your bathroom you don't call him, you call a plumber. The plumber may not be a poet at all, but the plumber is relevant there. You don't call Albert Einstein -- he may be a great physicist, but what does he know about plumbing?
