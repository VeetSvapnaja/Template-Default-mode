\chapter{Nyala Api Tanpa Asap} %179

\bahasa
25 Februari 1979 pagi di Aula Buddha

\english
25 February 1979 am in Buddha Hall

\bahasa
Pertanyaan Pertama:

\english
The first question:

\bahasa
OSHO TERKASIH,
ENGKAU TENTU SAJA SUDAH BERKALI-KALI MEMBERI TAHU KAMI, TAPI AKU TETAP TIDAK MENGERTI: APA BENIH DARI KEINGINAN? APAKAH ITU HANYA DALAM KEBERADAAN PIKIRAN? DAN BAGAIMANA KEINGINAN TUBUH AKAN SEKS BERHUBUNGAN DENGAN PIKIRAN?


\english
BELOVED OSHO,
YOU MUST HAVE TOLD US SO MANY TIMES ALREADY, BUT I STILL DON'T GET IT: WHAT IS THE SEED OF DESIRE? IS IT ONLY IN THE EXISTENCE OF MIND? AND HOW IS THE DESIRE OF THE BODY FOR SEX RELATED TO THE MIND?

\bahasa
Anurag, energi yang disebut keinginan telah dikutuk selama berabad-abad. Hampir semua yang disebut orang-orang suci menentangnya, karena keinginan adalah kehidupan dan orang-orang suci itu negatif. Keinginan adalah sumber semua yang engkau lihat, dan mereka menentang semua hal yang terlihat. Mereka ingin mengorbankan yang terlihat di kaki yang tak terlihat; Mereka ingin memotong akar dari keinginan sehingga tidak ada lagi kemungkinan hidup.

\english
Anurag, the energy called desire has been condemned for centuries. Almost all the so called saints have been against it, because desire is life and they were all life-negative. Desire is the very source of all that you see, and they were against all that which is visible. They wanted to sacrifice the visible at the feet of the invisible; they wanted to cut the roots of desire so there would no longer be any possibility of life.

\bahasa
Dorongan yang sangat besar untuk melakukan bunuh diri sepenuhnya telah mendominasi umat manusia selama berabad-abad.

\english
A tremendously great urge to commit total suicide has dominated humanity down the ages.

\bahasa
Aku memiliki konsep tentang keinginan yang sama sekali berbeda. Pertama, keinginan itu sendiri adalah Tuhan. Keinginan tanpa objek apapun, keinginan tanpa berorientasi pada tujuan, keinginan yang tidak termotivasi, keinginan murni, adalah Tuhan. Energi yang disebut keinginan adalah energi yang sama dengan Tuhan.

\english
I have a totally different concept of desire. First, desire itself is God. Desire without any object, desire without being goal-oriented, unmotivated desire, pure desire, is God. The energy called desire is the same energy as God.

\bahasa
Keinginan tidak untuk dihancurkan, itu harus dimurnikan. Keinginan tidak untuk dijatuhkan, itu harus diubah. Keberadaanmu adalah keinginan; untuk melawannya adalah melawan diri sendiri dan melawan semua. Melawannya adalah melawan bunga-bunga dan burung-burung, matahari dan bulan. Untuk melawannya adalah melawan semua kreativitas. Keinginan adalah kreativitas.

\english
Desire has not to be destroyed, it has to be purified. Desire has not to be dropped, it has to be transformed. Your very being is desire; to be against it is to be against yourself and against all. To be against it is to be against the flowers and the birds and the sun and the moon. To be against it is against all creativity. Desire is creativity.

\bahasa
Kitab suci dari Timur benar ketika mereka mengatakan bahwa Tuhan menciptakan dunia karena keinginan yang besar muncul di dalam dirinya - keinginan untuk mencipta, keinginan untuk mewujudkan, keinginan untuk membuat banyak dari satu, keinginan untuk berkembang. Tapi ini hanya metafora; Tuhan tidak terpisah dari keinginan. Keinginan berarti kerinduan, kerinduan yang besar, untuk berkembang, menjadi besar, menjadi besar sekali - sama besarnya dengan langit.

\english
The Eastern scriptures are perfectly right when they say that God created the world because a great desire arose in him -- a desire to create, a desire to manifest, a desire to make many from one, a desire to expand. But these are only metaphors; God is not separate from desire. Desire means a longing, a great longing, to expand, to become huge, to be enormous -- as huge as the sky.

\bahasa
Cukup menyaksikan orang-orang, menyaksikan keinginan, dan engkau akan mengerti apa yang aku maksud. Bahkan di dalam keinginan-keinginan biasamu, hal mendasar ada. Sebenarnya apa yang diinginkan oleh orang untuk memiliki lebih banyak dan lebih banyak uang apa yang benar-benar mereka inginkan bukanlah uang tapi perluasan, karena uang bisa membantumu untuk berkembang. Engkau bisa memiliki rumah yang lebih besar, engkau bisa memiliki taman yang lebih besar, engkau bisa memiliki ini, engkau bisa memiliki itu - wilayahmu akan lebih besar, kebebasanmu akan menjadi lebih besar. Dengan lebih banyak uang engkau akan memiliki lebih banyak alternatif untuk dipilih.

\english
Just watch people, watch desires, and you will understand what I mean. Even in your ordinary desires, the basic thing is present. In fact what the man who wants to have more and more money really wants is not money but expansion, because money can help you expand. You can have a bigger house, you can have a bigger garden, you can have this, you can have that -- your territory will be bigger, your freedom will be bigger. With more money you will have more alternatives to choose from.

\bahasa
Orang yang mengejar uang mungkin tidak tahu mengapa dia mengejar uang. Dia mungkin sendiri berpikir dan percaya bahwa dia mencintai uang, tapi itu hanya di permukaan kesadarannya. Pergilah lebih dalam ke bawah sadarnya, bantulah dia untuk bermeditasi, dan engkau akan terkejut dan dia akan terkejut untuk mengetahui bahwa keinginan untuk uang sebenarnya bukan keinginan untuk uang, itu adalah keinginan untuk memperluas.

\english
The man who is after money may not know why he is after the money. He may himself think and believe that he loves money, but that is only on the surface of his consciousness. Go deeper into his unconscious, help him to meditate, and you will be surprised and he will be surprised to find that the desire for money is not really the desire for money, it is the desire to expand.

\bahasa
Dan hal yang sama terjadi pada semua keinginan lainnya. Manusia menginginkan lebih banyak kekuatan, lebih terkenal, hidup lebih lama, kesehatan lebih baik, tapi apa yang mereka inginkan dari hal-hal yang berbeda ini? Hal yang sama, persis sama: mereka ingin menjadi lebih. Mereka tidak ingin tetap terkungkung, mereka tidak mau terbatas. Itu terasa sakit bahwa engkau bisa didefinisikan, karena jika engkau bisa didefinisikan maka engkau hanyalah sebuah objek, sebuah benda, sebuah komoditi. Itu sakit bahwa engkau punya keterbatasan, karena punya keterbatasan berarti dipenjara.

\english
And the same is the case with all other desires. Men want more power, more fame, longer life, better health, but what are they desiring in these different things? The same, exactly the same: they want to be more. They don't want to remain confined, they don't want to be limited. It hurts to feel that you are definable, because if you are definable then you are just an object, a thing, a commodity. It hurts that you have limitations, because to have limitations means to be imprisoned.

\bahasa
Tapi semua obyek-obyek keinginan ini, cepat atau lambat, mengecewakan. Uang menjadi mungkin suatu hari nanti, namun perluasan belum terjadi; engkau mungkin memiliki sedikit kebebasan untuk memilih, tapi itu tidak memuaskan. Keinginan adalah untuk yang tak terbatas, dan uang tidak bisa membeli yang tak terbatas. Ya, engkau memiliki lebih banyak kekuatan, engkau lebih terkenal, tapi itu tidak terlalu penting dalam jangka panjang. Jutaan orang telah hidup di bumi ini dan sangat terkenal, dan sekarang tidak ada yang tahu nama mereka. Segalanya telah hilang menjadi debu -- debu menjadi debu, bahkan jejak pun tidak ada lagi. Dimana Alexander yang Agung? Siapa dia? Apakah engkau ingin menjadi Alexander Agung yang telah meninggal atau seorang pengemis yang hidup? Tanyakan pada diri sendiri, dan keberadaanmu akan mengatakan lebih baik hidup dan menjadi pengemis daripada harus mati dan menjadi seorang Alexander.

\english
But all these objects of desire, sooner or later, disappoint. Money becomes possible one day, and yet expansion has not happened; you may have a little more freedom of choice, but that does not satisfy. The desire was for the infinite, and money cannot purchase the infinite. Yes, you have more power, you are more well-known, but that doesn't really matter in the long run. Millions of people have lived on this earth and were very famous, and now nobody even knows their names. Everything has disappeared into dust -- dust into dust, not even traces are left. Where is Alexander the Great? What is he? Would you like to be a dead Alexander the Great or an alive beggar? Ask yourself, and your being will say it is better to be alive and be a beggar than to be dead and be an Alexander.

\bahasa
Jika engkau memperhatikan dengan cermat, uang, kekuatan, kehormatan -- tidak ada yang memuaskan. Sebaliknya, mereka membuatmu lebih tidak puas. Mengapa? -- karena ketika engkau miskin ada harapan suatu hari nanti uang itu akan terjadi dan semua akan tenang dan tenang selamanya, dan kemudian engkau akan rileks dan menikmati. Sekarang itu telah terjadi, dan tampaknya tidak ada tanda-tanda relaksasi apapun. Sebenarnya, engkau lebih tegang dari sebelumnya, engkau lebih cemas daripada sebelumnya.

\english
If you watch carefully, money, power, prestige -- nothing satisfies. On the contrary, they make you more discontented. Why? -- because when you were poor there was a hope that one day the money was going to happen and all would be settled and settled forever, and then you would relax and enjoy. Now that has happened, and there seems to be no sign of any relaxation. In fact, you are more tense than before, you are more anxiety-ridden than before.

\bahasa
Uang telah membawa beberapa berkah, tapi dalam ukuran yang sama juga telah membawa banyak kutukan. Engkau bisa memiliki rumah yang lebih besar, tapi sekarang engkau akan memiliki sedikit kedamaian. Engkau bisa memiliki saldo bank yang lebih besar, tapi engaku juga akan memiliki kegilaan, kegelisahan, neurosis, psikosis yang lebih besar. Uang telah membawa beberapa hal yang baik; di balik semuanya itu banyak hal lain telah tiba yang tidak baik sama sekali. Dan jika engkau melihat keseluruhannya, keseluruhan usaha itu hanyalah pemborosan belaka. Dan sekarang engkau bahkan tidak dapat memiliki harapan yang orang miskin dapat memilikinya.

\english
Money has brought a few blessings, but in the same measure it has brought many curses too. You can have a bigger house, but now you will have less peace. You can have a bigger bank balance, but you will also have a bigger madness, anxiety, neurosis, psychosis. Money has brought a few things which are good; in the wake of it many other things have arrived which are not good at all. And if you look at the whole thing, the whole effort has been a sheer wastage. And now you cannot have even the hope that the poor man can have.

\bahasa
Orang kaya menjadi putus asa. Dia tahu sekarang uangnya akan terus meningkat dan tidak ada yang akan terjadi -- cuma kematian, hanya kematian. Dia telah merasakan segala macam hal; sekarang dia hanya merasakan yang tidak berasa. Suatu jenis kematian sudah terjadi, karena dia tidak dapat memikirkan bagaimana memenuhi keinginan untuk perluasan.

\english
The rich man becomes hopeless. He knows now the money will go on increasing and
nothing is going to happen -- just death, only death. He has tasted all kinds of things; now he only feels a tastelessness. A kind of death has already happened, because he cannot conceive of how to fulfill that desire for expansion.

\bahasa
Tapi keinginan itu sendiri tidak salah. Keinginan akan uang, keinginan akan kekuasaan, keinginan akan kehormatan, adalah objek yang salah untuk keinginan - biarkan itu menjadi jelas. Dengan memiliki objek keinginan yang salah, keinginan itu sendiri tidak menjadi salah. Engkau bisa memiliki pedang dan engaku bisa membunuh seseorang -- itu tidak membuat pedang salah. Engkau juga bisa menyelamatkan seseorang dengan pedang yang sama. Racun bisa membunuh dan racun bisa menjadi obat juga. Di tangan yang benar, racun adalah nektar; di tangan yang salah, nektar adalah racun.

\english
But desire in itself is not wrong. The desire for money, the desire for power, the desire for prestige, are wrong objects for desire -- let it be very clear. By having wrong objects of desire, desire itself does not become wrong. You can have a sword and you can kill somebody -- that does not make the sword something wrong. You can also save somebody with the same sword. Poison can kill and poison can become medicine too. In the right hands, poison is nectar; in the wrong hands, nectar is poison.

\bahasa
Inilah kebijaksanaan penting para Buddha selama berabad-abad. Apa yang para imam agama katakan adalah satu hal; apa yang telah dibawa oleh para Buddha ke dunia ini sangat berbeda, hal itu sama sekali bertentangan.

\english
This is the essential wisdom of all the buddhas of all the ages. What the priests say is one thing; what the buddhas have brought to the world is totally different, it is diametrically opposite.

\bahasa
Keinginan harus dimurnikan dan diubah, karena itu energimu -- engkau tidak memiliki energi lain. Bagaimana mengubah keinginan? Salah satu cara, cara yang umum, cara yang biasa-biasa saja, adalah mengubah objek. Jangan mengejar uang, mulailah mengejar Tuhan. Engkau frustrasi dengan uang -- menjadi religius, pergi ke gereja, ke kuil, ke masjid. Biarkan keinginanmu memiliki objek baru yang disebut Tuhan, yang sama ilusinya dengan objek yang disebut uang, bahkan lebih ilusi lagi, karena apa yang engaku ketahui tentang Tuhan? Uang setidaknya adalah sesuatu yang terlihat, objektif; engkau sudah mengetahuinya, engkau sudah melihatnya. Apa yang engkau ketahui tentang Tuhan? Engkau hanya mendengar kata itu. Tuhan tetaplah sebuah kata kecuali jika dialami. Tuhan tetap kata yang kosong kecuali jika engkau menuangkan beberapa isi ke dalamnya melalui pengalaman nyatamu sendiri.

\english
Desire has to be purified and transformed, because it is your energy -- you don't have any other energy. How to transform desire? One way, the ordinary way, the mediocre way, is to change the object. Don't go after money, start going after God. You are frustrated with money -- become religious, go to the church, to the temple, to the mosque. Let your desire have a new object called God, which is as illusory as the object called money, even more illusory, because what do you know about God? Money at least is something visible, objective; you have known it, you have seen it. What do you know of God? You have only heard the word. God remains a word unless experienced. God remains an empty word unless you pour some content into it through your own existential experience.

\bahasa
Orang-orang, ketika mereka frustrasi dengan keinginan duniawi, mulai mengubah objek: mereka mulai membuat objek keinginan duniawi lainnya -- surga, firdaus, dan semua kegembiraan di surga. Tapi itu trik yang sama, pikiran lagi lagi membodohimu. Ini bukan cara orang cerdas, inilah cara orang bodoh.

\english
People, when they are frustrated with worldly desires, start changing the object: they start making other worldly objects of desire -- heaven, paradise, and all the joys of heaven. But it is the same trick, the mind is again befooling you. This is not the way of the intelligent person, this is the way of the stupid.

\bahasa
Apa itu kecerdasan? Kecerdasan berarti pengertaian yang mendalam bahwa tidak ada objek yang bisa memenuhi keinginanmu. Tidak ada objek, kataku, dan kukatakan itu secara pasti, tidak ada objek yang bisa memenuhi keinginanmu. Keinginanmu adalah ilahi. Keinginanmu sebesar langit -- bahkan langit tidak membatasinya. Tidak ada objek yang bisa mengisinya. Lalu apa yang harus dilakukan? Orang yang cerdas berhenti menginginkan objek. Dia membuat keinginannya murni dari semua objek -- duniawi, dunia lain. Dia mulai menjalani keinginannya dalam kemurniannya, tiap tiap momen. Dia penuh dengan keinginan, penuh dengan energi yang melimpah. Kehidupannya yang biasa menjadi begitu kuat, sangat bergairah, sehingga apapun yang disentuhnya akan diubahkan. Kotoran logam akan menjadi emas, dan pohon yang mati akan kembali mekar.

\english
What is intelligence? Intelligence means the insight that no object can fulfill your desire. No object, I say, and I say it categorically, no object can ever fulfill your desire. Your desire is divine. Your desire is as big as the sky -- even the sky is not a limit to it. No object can fill it. Then what is to be done? The intelligent person stops desiring objects. He makes his desire pure of all objects -- worldly, otherworldly. He starts living his desire in its purity, moment to moment. He is full of desire, full of overflowing energy. His ordinary life becomes so intense, so passionate, that whatsoever he touches will be transformed. The baser metal will become gold, and the dead tree will come to bloom again.

\bahasa
Dikatakan tentang Buddha bahwa kemanapun ia bergerak, pohon mati akan mulai tumbuh dedaunan; di luar musimnya, pepohonan akan mekar. Ini adalah ungkapan puitis yang indah dari kebenaran metafisik yang pasti. Buddha adalah keinginan murni, hanya keinginan. Bukan keinginan untuk apapun; dia telah meninggalkan semua objek.

\english
It is said of Buddha that wherever he moved, dead trees would start growing leaves; out of season, trees would bloom. These are beautiful poetic expressions of a certain metaphysical truth. Buddha is pure desire, just desire. Not a desire for anything; he has abandoned all objects.

\bahasa
Izinkan aku mengingatkanmu, pertama dia meninggalkan dunia. Dia adalah seorang pangeran, dia terlahir untuk menjadi raja. Melihat kesia-siaan uang, melihat kesia-siaan semua jenis hubungan, melihat kesia-siaan yang bisa diberikan oleh dunia -- dia baru berusia dua puluh sembilan tahun - dia melarikan diri. Dia melakukannya dengan baik, karena setelah tiga puluh tahun menjadi lebih sulit, semakin sulit.

\english
Let me remind you, first he abandoned the world. He was a prince, he was born to be a king. Seeing the futility of money, seeing the futility of all kinds of relationships, seeing the futility of all that the world can give -- he was only twenty-nine years old -- he escaped. He did well, because after thirty it becomes more difficult, more and more difficult.

\bahasa
Para hippy benar. Mereka berkata, "Jangan percaya seorang manusia berusia di atas tiga puluh tahun". Buddha melarikan diri pada waktu yang tepat -- tepatnya dia berumur dua puluh sembilan -- karena semakin engkau mengalami hal-hal duniawi, semakin engkau menjadi pengecut. Agama adalah untuk yang berani, agama adalah untuk para pemberani, agama bagi kaum muda, mereka yang masih mampu mengambil risiko, mereka yang masih mampu untuk berjudi.

\english
Hippies are right. They say, "Don't believe a man who is over thirty." Buddha escaped at the right time -- he was exactly twenty-nine -- because the more you become experienced in worldly ways, the more cowardly you become. Religion is for the courageous, religion is for the brave, religion is for the young, those who are still able to take the risk, those who are still able to gamble.

\bahasa
Buddha melarikan diri. Melihat kesia-siaan, dia melarikan diri untuk mencari Tuhan, untuk mencari kebenaran. Dia menggantikan keinginannya akan dunia dengan keinginan akan Tuhan, kebenaran, nirwana. Selama enam tahun ia bekerja keras. Pada saat berusia tiga puluh lima, dia benar-benar letih. Dia telah melakukan semua yang mungkin, mungkin secara manusiawi lakukan. Dia berpuasa selama berbulan-bulan, bermeditasi, berlatih yoga. Dan pada masa itu ada berbagai jenis sekolah. Dia pergi dari satu guru ke guru lainnya, dari satu sekolah ke sekolah lainnya, dia mempraktekkan semua metode yang mungkin. Dan suatu hari muncul dengan tiba-tiba.
