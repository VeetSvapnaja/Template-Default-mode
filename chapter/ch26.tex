\chapter{Elektron yang tidak logis} %334

\bahasa
8 Maret 1979 pagi di Aula Buddha

\english
8 March 1979 am in Buddha Hall

\bahasa
Pertanyaan pertama:

\english
The first question:

\bahasa
Osho Terkasih,\\
KEBEBASAN DAPAT DICAPAI MELALUI KEHENDAK, CINTA TIDAK. TOLONG BERI KOMENTAR.

\english
BELOVED OSHO,\\
FREEDOM CAN BE WILLED, LOVE NOT. PLEASE COMMENT.

\bahasa
Anand Akam, kebebasan dapat dicapai melalui kehendak karena kebebasan adalah keputusanmu sendiri untuk tetap berada di sebuah penjara. Itu adalah tanggung jawabmu sendiri. Engkau telah menghendaki perbudakanmu, engkau telah memutuskan untuk tetap menjadi seorang budak, maka engkau adalah seorang budak. Ubahlah keputusan itu, dan perbudakan lenyap.

\english
Anand Akam, freedom can be willed because it is your own decision to remain in a
prison. It is your own responsibility. You have willed your slavery, you have decided to remain a slave, hence you are a slave. Change the decision, and the slavery disappears.

\bahasa
Engkau telah berinvestasi dalam ketidakbebasanmu. Setiap momen engkau melihat intinya, engkau dapat menanggalkannya; itu langsung bisa ditanggalkan. Tidak ada yang memaksakan ketidakbebasan kepadamu, itu adalah pilihanmu. Engkau dapat memilih untuk bebas, engkau dapat memilih untuk tidak bebas; engkau sangat bebas sehingga engkau dapat memilih salah satunya. Ini adalah bagian dari kebebasan batinmu- tidak memilihnya adalah bagian dari kebebasanmu. Makanya itu dapat dicapai melalui kehendak.

\english
You have invested in your unfreedom. Any moment you see the point, you can drop it; instantly it can be dropped. Nobody has forced unfreedom on you, it is your choice. You can choose to be free, you can choose to be unfree; you are so free that you can choose either. This is part of your inner freedom -- not to choose it is part of your freedom. Hence it can be willed.

\bahasa
Tapi cinta tidak dapat dicapai melalui kehendak. Cinta adalah produk sampingan dari kebebasan; cinta adalah sukacita yang melimpah dari kebebasan, cinta adalah aroma kebebasan. Pertama, kebebasan harus ada di sana, lalu cinta mengikuti. Jika engkau mencoba untuk mencintai, engkau hanya akan menciptakan sesuatu yang tidaklah alami, buatan. Cinta yang dicapai melalui kehendak tidak akan menjadi cinta sejati, itu akan menjadi kepalsuan.

\english
But love cannot be willed. Love is a by-product of freedom; it is the overflowing joy of freedom, it is the fragrance of freedom. First the freedom has to be there, then love follows. If you try to will love, you will create only something artificial, arbitrary. A willed love will not be true love, it will be phony.

