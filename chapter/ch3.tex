\chapter{Sitnalta dan Tujuhbelas Chakra}

\bahasa
Pertanyaan 1\\
OSHO Terkasih, Kenapa Aku tidak mempercayaimu?

\english
Question 1\\
BELOVED OSHO, WHY DON'T I TRUST YOU?

\bahasa
Prem Prageeta, kepercayaan hanya mungkin jika pertama-tama engkau percaya pada dirimu sendiri. Hal yang paling mendasar harus terjadi di dalam dirimu terlebih dahulu. Jika engkau mempercayai diri sendiri, engkau dapat mempercayaiku, engkau dapat percaya pada orang lain, engkau dapat percaya pada Semesta. Tapi jika engkau tidak percaya pada diri sendiri maka tidak ada kepercayaan lain yang mungkin terjadi.

\english
Prem Prageeta, trust is possible only if first you trust in yourself. The most fundamental thing has to happen within you first. If you trust in yourself you can trust in me, you can trust in people, you can trust in existence. But if you don't trust in yourself then no other trust is ever possible.

\bahasa
Dan masyarakat menghancurkan kepercayaan di akar-akarnya. Masyarakat tidak mengizinkanmu untuk mempercayai diri sendiri. Masyarakat mengajarkan semua jenis kepercayaan lainnya - kepercayaan pada orang tua, kepercayaan kepada gereja, kepercayaan kepada negara, kepercayaan kepada Tuhan, tiada habis-habisnya. Tapi kepercayaan dasar benar-benar dihancurkan . Dan kemudian semua kepercayaan lainnya menjadi kepalsunya, secara alami menjadi palsu. Kemudian semua kepercayaan lainnya hanyalah bunga plastik. Engkau tidak memiliki akar nyata supaya bunga nyata dapat tumbuh.

\english
And the society destroys trust at the very roots. It does not allow you to trust yourself. It teaches all other kinds of trust -- trust in the parents, trust in the church, trust in the state, trust in God, ad infinitum. But the basic trust is completely destroyed. And then all other trusts are phony, are bound to be phony. Then all other trusts are just plastic flowers. You don't have real roots for real flowers to grow.

\bahasa
Masyarakat melakukannya dengan sengaja, dengan terencana, karena seseorang yang mempercayai dirinya berbahaya bagi masyarakat - sebuah masyarakat yang bergantung pada perbudakan, masyarakat yang telah menginvestasikan terlalu banyak dalam perbudakan.

\english
The society does it deliberately, on purpose, because a man who trusts in himself is dangerous for the society -- a society that depends on slavery, a society that has invested too much in slavery.

\bahasa
Seseorang yang percaya dirinya sendiri adalah orang yang bebas. Engkau tidak dapat membuat prediksi tentang dia, dia akan bergerak dengan caranya sendiri. Kebebasan akan menjadi hidupnya. Dia akan percaya saat dia merasa, saat dia mencintai, dan kemudian kepercayaannya akan memiliki intensitas dan kebenaran yang luar biasa di dalamnya. Kemudian kepercayaannya akan hidup dan otentik. Dan dia akan siap mengambil risiko untuk kepercayaannya, tapi hanya saat dia merasakan hal itu, hanya jika memang benar, hanya saat hal itu menggerakkan hatinya, hanya saat hal itu menggerakkan kecerdasan dan cintanya, sebaliknya tidak. Engkau tidak dapat memaksanya untuk meyakini apa pun.

\english
A man trusting himself is an independent man. You cannot make predictions about him, he will move in his own way. Freedom will be his life. He will trust when he feels, when he loves, and then his trust will have a tremendous intensity and truth in it. Then his trust will be alive and authentic. And he will be ready to risk all for his trust, but only when he feels it, only when it is true, only when it stirs his heart, only when it stirs his intelligence and his love, otherwise not. You cannot force him into any kind of believing.

\bahasa
Dan masyarakat ini tergantung pada keyakinan. Seluruh strukturnya adalah autohypnosis. Seluruh strukturnya berbasis dalam menciptakan robot dan mesin, bukan manusia. Dibutuhkan orang-orang yang ketergantungan - begitu banyak sehingga mereka terus-menerus membutuhkan tirani, begitu banyak sehingga begituMencari dan mencari tirani mereka sendiri, Adolf Hitler mereka sendiri, Mussolinis mereka sendiri, Stalins Josef dan Mao Zedong mereka sendiri.

\english
And this society depends on belief. Its whole structure is that of autohypnosis. Its whole structure is based in creating robots and machines, not men. It needs dependent people -- so much so that they are constantly in need of being tyrannized, so much so that they are searching and seeking their own tyrants, their own Adolf Hitlers, their own Mussolinis, their own Josef Stalins and Mao Zedongs.

\bahasa
Bumi ini, bumi yang indah ini, kita telah mengubahnya menjadi penjara besar. Beberapa orang yang memiliki nafsu-kekuasaan telah menjadikan keseluruhan umat manusia sebagai gerombolan. Manusia dibiarkan ada hanya jika dia berkompromi dengan segala macam omong kosong.

\english
This earth, this beautiful earth, we have turned into a great prison. A few power-lusty people have reduced the whole of humanity into a mob. Man is allowed to exist only if he compromises with all kinds of nonsense.

\bahasa
Sekarang, untuk memberi tahu seorang anak untuk percaya kepada Tuhan adalah omong kosong, omong kosong belaka - bukan berarti Tuhan tidak ada, tapi karena anak belum merasakan kehausan, keinginan, kerinduannya. Dia belum siap untuk mencari kebenaran, kebenaran tertinggi kehidupan. Dia belum cukup dewasa untuk menyelidiki realitas Tuhan. Hubungan cinta itu harus terjadi suatu hari nanti, tapi itu dapat terjadi hanya jika tidak ada keyakinan yang dipaksakan padanya. Jika dia dirubah sebelum kehausan telah muncul untuk mengeksplorasi dan untuk mengetahui, maka seluruh hidupnya dia akan hidup dengan cara yang tidak sungguh-sungguh, dia akan hidup dengan cara yang palsu.

\english
Now, to tell a child to believe in God is nonsense, utter nonsense -- not that God does not exist, but because the child has not yet felt the thirst, the desire, the longing. He is not yet ready to go in search of the truth, the ultimate truth of life. He is not yet mature enough to inquire into the reality of God. That love affair has to happen some day, but it can happen only if no belief is imposed upon him. If he is converted before the thirst has arisen to explore and to know, then his whole life he will live in a phony way, he will live in a pseudo way.

\bahasa
Ya, dia akan berbicara tentang Tuhan, karena dia telah diberitahu bahwa Tuhan itu ada. Dan dia telah diberi tahu secara otoritatif, dan dia telah diberitahu oleh orang-orang yang sangat berkuasa di masa kecilnya - orang tuanya, para imam, para guru. Dia telah diberitahu oleh orang-orang itu dan dia harus menerimanya; Itu adalah pertanyaan tentang kelangsungan hidupnya. Dia tidak bisa mengatakan tidak pada orang tuanya, karena tanpa mereka dia tidak akan bisa hidup sama sekali. Terlalu berisiko mengatakan tidak, dia harus mengatakan ya. Tapi ya-nya dia tidak benar.

\english
Yes, he will talk about God, because he has been told that God is. And he has been told authoritatively, and he has been told by people who were very powerful in his childhood -  his parents, the priests, the teachers. He has been told by people and he had to accept it; it was a question of his survival. He could not say no to his parents, because without them he would not be able to live at all. It was too risky to say no, he had to say yes. But his yes can't be true.

\bahasa
Bagaimana itu bisa benar? Dia mengatakan ya hanya sebagai alat politik, untuk bertahan hidup. Engkau belum mengubahnya menjadi orang religius, engkau telah menjadikannya seorang diplomat, engkau telah menciptakan seorang politikus. Engkau telah merusak potensinya untuk tumbuh menjadi makhluk yang otentik. Engkau telah meracuni dia. Engkau telah menghancurkan kemungkinan kecerdasannya, karena kecerdasan muncul hanya ketika kerinduan muncul untuk mengetahui.

\english
How can it be true? He is saying yes only as a political device, to survive. You have not turned him into a religious person, you have made him a diplomat, you have created a politician. You have sabotaged his potential to grow into an authentic being. You have poisoned him. You have destroyed the very possibility of his intelligence, because intelligence arises only when the longing arises to know.

\bahasa






