\chapter{Menabur Benih Putih} %54

\bahasa
15 Februari 1979 pagi di Aula Buddha

\english
15 February 1979 am in Buddha Hall

\bahasa
MULAILAH PENGEMBANGAN UNTUK MENANGGUNG DIRIMU SENDIRI

\english
BEGIN THE DEVELOPMENT OF TAKING WITH YOURSELF. \\
WHEN EVIL FILLS THE INANIMATE AND ANIMATE UNIVERSES CHANGE \\
BAD CONDITIONS TO THE BODHI PATH. \\
DRIVE ALL BLAME INTO ONE. \\
BE GRATEFUL TO EVERYONE. \\
THE INSURPASSABLE PROTECTION OF EMPTINESS IS TO SEE THE \\
MANIFESTATIONS OF BEWILDERMENT AS THE FOUR KAYAS. \\
AN EXCELLENT MEANS IS TO HAVE THE FOUR PROVISIONS. \\
IN ORDER TO BRING ANY SITUATION TO THE PATH QUICKLY AS SOON AS \\
IT IS MET, JOIN IT WITH MEDITATION. \\
THE CONCISE EPITOME OF HEART INSTRUCTION: WORK WITH "FIVE FORCES." \\
THE INSTRUCTIONS FOR TRANSFERENCE IN THE MAHAYANA ARE THE "FIVE FORCES." \\
BEHAVIOR IS IMPORTANT. \\
THE PURPOSE OF ALL DHARMA IS CONTAINED IN ONE POINT \\

\bahasa
Meditasi adalah sumber, welas asih adalah luapan sumber itu. Manusia tak-meditatif tidak memiliki energi untuk cinta, untuk welas asih, untuk perayaan. Manusia tak-meditatif terputus dari sumber energinya sendiri; dia tidak terhubung dengan samudera. Dia memiliki sedikit energi yang diciptakan oleh makanan, oleh udara, oleh materi - dia hidup dengan energi fisik. Energi fisik memiliki keterbatasan. Energi itu lahir pada saat tertentu pada waktunya, dan hilang pada saat lain pada waktunya. Antara kelahiran dan kematian energi itu ada. Analoginya seperti sebuah lampu yang terbakar karena minyak di dalamnya - setelah minyak habis, nyala api padam.

\english
Meditation is the source, compassion is the overflow of that source. The nonmeditative man has no energy for love, for compassion, for celebration. The nonmeditative person is disconnected from his own source of energy; he is not in contact with the ocean. He has a little bit of energy that is created by food, by air, by matter -- he lives on physical energy. Physical energy has limitations. It is born at a certain moment in time, and it dies at another moment in time. Between birth and death it exists. It is like a lamp that burns because of the oil in it -- once the oil is exhausted, the flame goes out.

\bahasa

\english

